\documentclass{article}
\usepackage[utf8]{inputenc}
\usepackage{multicol}
\usepackage{geometry}
\usepackage{xcolor}
\usepackage{pgffor}

% pandoc syntax highlighting

\fontfamily{sans-serif}
\selectfont

\geometry{letterpaper, top=60pt, left=80pt, right=120pt}
\setlength{\columnsep}{2cm}
\definecolor{light}{rgb}{0.56078431373, 0.52941176471 ,0.52941176471}

\begin{document}

  {\noindent \huge Chocolate con leche de coco}
  \vspace{1cm}

  \begin{multicols}{2}
  \noindent \color{light}
      8 pastillas de chocolate\newline
      2 tazas de zumo o primera leche de coco\newline
        1 litro de leche de vaca\newline
      azúcar a gusto.\newline
    \end{multicols}
  \vspace{1cm}

  {\noindent \LARGE Instruciones}\\
  \\
  \noindent \color{light} Se pone a cocinar a fuego lento la leche de coco con las pastillas de
chocolate, ba-tiendo fuertemente con un molinillo. Cuando estén
disueltas se les agrega la lechede vaca y el azú­car. Se bate bien hasta
que espese a gusto. Se sirve caliente.

\end{document}
